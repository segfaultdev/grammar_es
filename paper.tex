%% LyX 2.3.7 created this file.  For more info, see http://www.lyx.org/.
%% Do not edit unless you really know what you are doing.
\RequirePackage{fix-cm}
\documentclass[12pt]{article}
\PassOptionsToPackage{natbib=true}{biblatex}
\usepackage{amstext}
\usepackage{fontspec}
\setmainfont[Mapping=tex-text]{Serif}
\usepackage[a4paper]{geometry}
\geometry{verbose,tmargin=2.54cm,bmargin=2.54cm,lmargin=2.54cm,rmargin=2.54cm}
\setcounter{secnumdepth}{5}
\setcounter{tocdepth}{5}
\setlength{\parindent}{1cm}
\usepackage{verbatim}
\usepackage{setspace}
\setstretch{1.6}
\usepackage[unicode=true,pdfusetitle,
 bookmarks=true,bookmarksnumbered=false,bookmarksopen=false,
 breaklinks=false,pdfborder={0 0 1},backref=false,colorlinks=false]
 {hyperref}

\makeatletter
%%%%%%%%%%%%%%%%%%%%%%%%%%%%%% User specified LaTeX commands.
\usepackage{indentfirst}

\AtBeginDocument{
  \def\labelitemii{\(\circ\)}
  \def\labelitemiii{\(\star\)}
}

\makeatother

\usepackage{polyglossia}
\setdefaultlanguage{spanish}
\usepackage[bibstyle=apa,citestyle=numeric]{biblatex}
\addbibresource{paper.bib}
\begin{document}
\title{Modelo determinista para el análisis sintáctico: una perspectiva algorítmica
y computacional de la lingüística}
\author{Julio Meroño Sáez}
\maketitle
\begin{abstract}
(English version)

(Versión en español)
\end{abstract}
\tableofcontents{}

\part{Prólogo}

\section{Introducción}

El español o castellano es una lengua iberorromance y es la lengua
nativa de más de 450 millones de hablantes en el mundo, a lo que se
le suman los 75 millones de hablantes no nativos alrededor del planeta.
Esto la sitúa como la cuarta lengua más hablada globalmente, detrás
del inglés, el chino mandarín y el hindi. Es por ello que la existencia
de una lingüística estandarizada es fundamental, al permitir la comunicación
internacional sin cabida a ningún tipo de ambigüedad y al mantener
la integridad de la lengua. La Asociación de Academias de la Lengua
Española, junto a sus correspondientes academias nacionales (entre
ellas, la Real Academia Española), son las encargadas de dicho cometido,
proporcionando una estandarización de la gramática, morfología, léxico
y fonética del español alrededor del mundo hispanohablante.

\bigskip

Sin duda, la gramática es tan fundamental como las otras ramas de
estudio de la lingüística. Esta es la encargada de estudiar la organización
y la función de las palabras dentro de una oración, junto a la agrupación
de palabras en unidades mayores con significado y función, denominadas
sintagmas. A su vez, dichos sintagmas se organizan por categorías
sintagmáticas que engloban y marcan su función general. El estudio
y el estandarizado de la gramática en el español se remonta a finales
del siglo XV, con autores como Antonio de Nebrija, llegando y continuando
hasta la actualidad con teorías y propuestas transversales a la mayoría
de lenguas, como las propuestas por Noam Chomsky, o trabajos más enfocados
en el español.

\bigskip

Tradicionalmente, y aún en épocas recientes, se ha realizado un estudio
sintáctico más específico para cada categoría sintagmática, donde
los sintagmas (o grupos, como se hacen llamar en ciertos modelos)
poseen estructuras fundamentalmente distintas entre tipos, y donde
se da lugar a numerosos casos límite a tener en cuenta. Aún así, dicho
análisis ha resultado ser considerablemente más efectivo y simple
a la hora de su uso superficial por lingüistas procedentes de otras
ramas, o en el ámbito educativo como medio de enseñanza. Sin embargo,
desde finales del siglo anterior y gracias a la teoría estándar propuesta
por Noam Chomsky, se ha dado lugar a otro enfoque distinto a la gramática,
denominado gramática generativa. En la mayoría de teorías y modelos
asociados con dicha gramática, se formula que todas las agrupaciones
a las que se les denomina sintagmas comparten una estructura fundamental,
que aun pudiendo variar en el tipo de unidades y tipos de sintagmas
agrupados para formarlo, siempre se estructuran de la misma manera,
sin casi ninguna cabida a variación.

\bigskip

El estudio de la gramática en el español, especialmente de la sintaxis,
ha resultado en el creciente interés por el uso de las nuevas tecnologías
y de la capacidad computacional de este siglo para el análisis sintáctico
de frases, sintagmas individuales y oraciones. Por desgracia, el uso
en su mayoría de la gramática tradicional para la creación de modelos
y algoritmos que cumplan dicha tarea dificulta considerablemente la
detección y el correcto análisis de ciertos casos especiales o límite
que se pueden dar, o imposibilita casi por completo el proporcionamiento
de más información en cuanto a la función que realizan ciertos sintagmas
dentro de agrupaciones mayores, factor que la gramática tradicional
no suele tener en cuenta.

\bigskip

Es por ello por lo que este trabajo de investigación propone un análisis
determinista en forma de modelo adaptado de la gramática del español,
basándose fundamentalmente en la gramática generativista aportada
en la obra de Karen Zagona, Sintaxis generativa del español {[}1{]},
pero incluyendo modificaciones tanto propias como de otros autores
que permitan una mayor facilidad para el procesado de oraciones mediante
implementaciones computacionales de dicho modelo. Además, se pretende
diseñar y llevar a cabo una implementación tipo del modelo formulado
previamente que sea capaz de poder utilizarse en ordenadores convencionales
sin requerir de un excesivo coste computacional, para su posible y
posterior publicación mediante una interfaz en línea y de acceso libre
para su uso público.

\section{Justificación}

Son decenas de millones los estudiantes que se encuentran actualmente
aprendiendo español en las escuelas. En específico, contando únicamente
los estudiantes extranjeros y no procedentes de ningún país hispanohablante,
se estima que el número se encuentra en el orden de los 24 millones
{[}2{]}. Entre toda la materia y contenidos aportados en la clase
a dichos estudiantes durante sus horas lectivas, una gran y significativa
parte consiste en el estudio de la propia lengua. Además, la utilización
del estudio y análisis sintáctico mediante separación en sintagmas
o grupos encuadrados se ha vuelto a lo largo de los años el estándar
a nivel nacional y, en ciertos casos, internacional. Aunque los estudios
que lo afirmen son relativamente escasos, predominantemente la experiencia
ha demostrado a las instituciones que planifican las competencias
a tener en cuenta a la hora de enseñar en las aulas que el entendimiento
del funcionamiento de la lengua implica un mejor uso de esta, tanto
social y oral como formal y escrito, y es fundamental para la preservación
del español, razón junto a las anteriores por las que se muestra un
interés significativo en su enseñanza en los centros educativos a
lo largo de todo el territorio español. Además, se ha de tener en
cuenta la reciente y creciente adopción de esta metodología en otros
países de habla española, lo que implica una mayor impulsión del conocimiento
interno del funcionamiento de la sintaxis española.

\bigskip

Si bien, el uso de la gramática tradicional como herramienta didáctica
se remonta a, aproximadamente, el siglo anterior, la cantidad de material
de ejercicios para los estudiantes es particularmente escaso, especialmente
cuando se pretende comparar con otras materias donde no solo existe
mucho más material, sino que es posible su generación procedural y
automática. En principio, lo comentado no debería tener ninguna implicación
negativa, pues al tratarse de un estudio de la gramática española
en forma de sintaxis basada en teorías tradicionales, la generación
de ejercicios resulta tan simple como obtener, buscar o producir por
la propia cuenta del profesorado oraciones que se adapten al nivel
de los estudiantes. Sin embargo, se ha de considerar y tener en cuenta
que, para comprobar la correctitud y validez de las respuestas otorgadas
por los estudiantes a espera de ser evaluadas, resulta tedioso tener
que realizar una comprobación mental lenta y laboriosa de dichos resultados,
que puede además dar lugar a numerosos errores de comprobación, no
solo beneficiosos sino también perjudiciales en ciertos casos para
el alumnado. Por otro lado, desde el punto de vista estudiantil, la
comprobación independiente de resultados es completamente imposible,
pues se carece de un punto de referencia objetivo para comparar, por
lo que estos carecen de una forma de poder practicar de manera efectiva
e ilimitada, sin depender del continuo trabajo del profesorado.

\bigskip

Lo que diferencia principalmente a las personas de cualquiera de los
dos puntos de vista anteriores con los ordenadores y dispositivos
electrónicos es la capacidad de poder tomar decisiones aplicando el
entendimiento horizontal, subjetivo y basado en la experiencia y no
en reglas a aplicar. Por lo tanto, la tarea de producir análisis sintácticos
utilizables en el ámbito de la creación de asistentes basados en la
inteligencia artificial se considera un tema popular y continuamente
en desarrollo y debate en investigaciones de las últimas décadas.

\bigskip

Este último caso nos lleva al ámbito judicial, donde se muestra una
carencia de estandarización intuitiva del significado de los textos
legislativos, los contratos y los acuerdos formales, dando lugar a
que existan profesiones cuyo único propósito sea el de garantizar
la ausencia de vacíos legales a la hora de producir dichos papeles.
Todo esto ocurre por una vaga especificación abierta a interpretaciones
de la gramática utilizada en este campo.

\bigskip

La motivación de este trabajo de investigación es, entonces, aportar
un pequeño pero significante grano de arena a la base de lo que será
el mundo en los años y las décadas por venir, tanto en el ámbito de
la educación, como en el de la tecnología, como en el judicial, como
en cualquier otro que se pueda plantear (lenguaje matemático, diplomacia,
etcétera). Para ello, se plantean diversos objetivos escalonados que
se formulan en el apartado a continuación.

\section{Objetivos del trabajo}

El objetivo principal de este trabajo de investigación es, en resumidas
cuentas, poder aportar un modelo descriptivo y determinista pero empírico
de la gramática del español, basándose en una agrupación de teorías
y definiciones de la gramática de autores anteriores y antecedentes,
mezclado con un enfoque propio matizado por la mentalidad computacional
y matemática, que pueda utilizarse como herramienta de formalización,
educación y comprobación, pero además como punto de partida para futuras
investigaciones.

\bigskip

Por otro lado, este objetivo ya mencionado da lugar a ciertos enfoques
alternativos pero compatibles por los que llevar la investigación,
en forma de objetivos secundarios:
\begin{enumerate}
\item Crear una herramienta de uso fácil e inmediato para el análisis de
oraciones y sintagmas en español, disponible de forma completamente
gratuita al público, para su uso por profesorado, estudiantes, políticos,
diplomáticos, abogados, etcétera.
\item Aportar una simplificación general de las estructuras de árbol presentes
en la mayoría de corrientes presentes, principalmente del árbol bipartito
propuesto por las principales ramas de la gramática generativista,
que permita su uso de forma más eficaz a la hora de realizar análisis
computacionales de la sintaxis del español, no solo como punto de
apoyo para el anterior objetivo sino también como base para modelos
de asistencia mediante inteligencia artificial, chatbots y cualquier
tecnología similar. %
\begin{comment}
\begin{enumerate}
\item TODO: ¿Reescribir?
\end{enumerate}
\end{comment}
\item Agrupar los esfuerzos provenientes de las distintas corrientes lingüísticas
existentes para propulsar la gramática española en un trabajo recopilatorio
que, además, introduce ciertas variaciones partiendo de un enfoque
más tecnológico, algorítmico y modélico. 
\end{enumerate}

\part{Modelo adaptado de la gramática española}

En esta parte del trabajo de investigación, se diseña y redacta un
modelo adaptado de la gramática del español, con la intención de su
posterior uso en la implementación del mismo mediante técnicas propias
del ámbito de la programación y la algoritmia. Previamente a ello,
sin embargo, se nombra la base a partir de la que se construye el
modelo, y se menciona en detalle la estructura, metodología de diseño
y paradigma utilizado para la creación del mismo. %
\begin{comment}
¿Qué se hace en esta parte? Breve mención de las secciones.
\end{comment}

\bigskip

Es de relevancia para el entendimiento del trabajo tener en cuenta
que, en la redacción del modelo, se parte de una base nula, posiblemente
reasignando definiciones incompatibles a cierta terminología en uso
con el fin de simplificar la producción y aumentar la claridad de
este. Además, las propuestas de clasificación, análisis y estructurado
pueden poseer disparidades significativas con las procedentes de las
corrientes gramaticales actuales, incluso, en definiciones generalmente
aceptadas previamente. Esto es debido, por otro lado, a los intentos
de acercar el modelo a una redacción lo suficientemente formal en
el sentido algorítmico como para su utilización sin posible lugar
a dudas en las implementaciones. %
\begin{comment}
Lenguaje utilizado y disparidades con la gramática actual.
\end{comment}

\bigskip

Por último, por limitaciones de tiempo, de personal y de antecedentes
previos a este trabajo de investigación, resulta considerablemente
dificil, incluso imposible en la práctica, tratar de definir la totalidad
de la gramática española en el mismo, por lo que se ha decidido reducir
la lengua en cuestión a un subconjunto del español. Con esto se pretende
decir, adicionalmente, que cualquier enunciado válido en esta versión
reducida del español es, también, un enunciado válido en el español
real. Esta reducción elimina ciertas estructuras complejas, como la
gran mayoría de construcciones oracionales o la yuxtaposición de cualquier
tipo. También se descartan ciertas variaciones internas dentro las
estructuras sintagmáticas, limitándolas a formas y ordenamientos fácilmente
procesables y comunes en el español, mencionados posteriormente en
el trabajo. %
\begin{comment}
Subconjunto del español en cuestión, omisión de construcciones y estructuras
complejas, etcétera.
\end{comment}


\section{Fuentes e investigaciones previas}

El modelo presentado y definido a lo largo de las siguientes secciones
obtiene su inspiración y punto de partida en diversas corrientes gramaticales
y lingüísticas existentes en el momento. Sin embargo, dichas corrientes
carecen del rigor matemático o algorítmico necesario para la realización
de implementaciones formalmente verificables. De esta manera, el marco
bibliográfico constituido por trabajos de relevancia para el diseño
del modelo se encuentra severamente limitado. Esto conlleva, como
se ha mencionado previamente, la necesidad de simplificar y reducir
la gramática tratada en el modelo a un conjunto reducido y delimitado
por lo fácilmente deducible y extrapolable a definiciones formales
de lo explicado en las fuentes bibliográficas e investigaciones previas
de este trabajo de investigación. Sin embargo, esto implica que las
fuentes en uso juegan un papel meramente inspiracional o de guía en
las definiciones aportadas, por lo que se ruega discreción a la hora
de su interpretación en las secciones posteriores.

\bigskip

A continuación, se mencionan brevemente las fuentes que se han utilizado
para la construcción y el diseño del modelo:
\begin{enumerate}
\item \textit{Corpus AnCora.} \citealt{antonia_marti_ancora_2007} Este
corpus español-catalán posee diversas anotaciones morfológicas, sintácticas
y semánticas de significativa utilidad para este trabajo de investigación.
En especial, los dos lexicones extraídos del mismo, AnCora-Verb y
AnCora-Nom, forman la base del procesado semántico efectuado en la
implementación del modelo. Las agrupaciones en argumentos de los roles
semánticos presentes en el modelo se realizan con la intención de
simplificar la tarea de llevar su análisis a la práctica, alineándolas
con las presentes en estos lexicones.
\item \textit{Nueva gramática de la lengua española.} \citealt{asociacion_de_academias_de_la_lengua_espanola_asale_real_academia_espanola_rae_nueva_2009}
Desde su publicación en 2009, esta obra académica ha servido como
actualización de la gramática normalizada por las mismas entidades
en la primera mitad del siglo anterior. En este trabajo de investigación,
la gramática procedente de la misma ha constituido la base para gran
parte de la terminología y definiciones en uso y la clasificación
de elementos, especialmente en los apartados más relacionados con
la morfología y la enumeración de rasgos gramaticales.
\item \textit{Corpus del español del siglo XXI.} \citealt{real_academia_espanola_rae_corpus_2023}
Las tablas de frecuencias de este corpus han sido de considerable
utilidad para la formación de la base de datos de elementos gramaticales
en uso en la posterior implementación, además de proporcionar un análisis
de rasgos gramaticales significativamente exhaustivo, razón por la
que esta fuente ha sido crucial para el trabajo de investigación.
Es por ello por lo que, para el diseño del modelo, se ha tomado la
leyenda de rasgos gramaticales proporcionados en el análisis como
punto de partida para su nombramiento y clasificación.
\item \textit{Sintaxis generativa del español.} \citealt{zagona_sintaxis_2006}
En este libro, se proporciona una visión global de la sintaxis del
español desde un punto de vista basado en la gramática generativista.
De esta visión surge gran parte de las estructuras de las cláusulas
y los sintagmas tratadas en este trabajo de investigación, además
de, en general, servir como inspiración movida por las corrientes
generativistas para la planificación y el diseño de ciertas secciones
del modelo, tanto en los apartados enfocados en la morfología como
en los que tratan principalmente la sintaxis del subconjunto del español
en cuestión.
\end{enumerate}
\begin{comment}
TODO: Añadir más fuentes aquí. Si se llegan a utilizar más, claro.
\end{comment}


\section{Estructura y visión del modelo}

Dentro del modelo redactado a continuación, se proporciona un enfoque
mayoritariamente funcional y empírico, evitando comprometer la simplicidad
del modelo que se perdería al pretender tratar temas como el origen
natural de ciertas estructuras. Por ello, se proporcionan declaraciones
que repliquen las construcciones naturales de la lengua con la mayor
precisión posible dentro de los límites prácticos del trabajo. %
\begin{comment}
Enfoque más funcional, o generativista, o lo que sea, hablar de eso
aquí.
\end{comment}

\bigskip

La estructura del mismo consiste en tres niveles fundamentales, ordenados
según su dependencia en los niveles anteriores y según el grado de
abstracción presente en cada uno de ellos. De esta manera, se comienza
tratando el nivel léxico-morfológico, donde se trata la base léxica
de la que se parte, sus rasgos gramaticales y la clasificación de
estos. A continuación, se menciona el nivel sintáctico, en el que
se explican las distintas estructuras válidas para la agrupación de
lexías en enunciados o subdivisiones de los mismos. Por último, se
describe el nivel semántico, cuya función es la de extraer y procesar
el significado de las estructuras formadas mediante la asignación
de roles. %
\begin{comment}
División entre nivel léxico-morfológico, nivel sintáctico y nivel
semántico.
\end{comment}

\bigskip

De cualquier manera que se plantease el trabajo, sin embargo, siempre
sería necesario partir de una base de conocimientos previos no tratados
en el mismo, por lo que, durante la creación del modelo, ciertos apartados
han sido omitidos por cuestiones de limitaciones o de conveniencia.
Específicamente, dos casos son inmediatamente perceptibles y de relevancia:
\begin{comment}
Omisión de apartados como la formación y flexión de palabras en morfología,
o la procedencia de los roles en semántica.
\end{comment}

\begin{enumerate}
\item El nivel léxico-morfológico presenta un vacío de contenido en cuanto
al origen, la formación y la flexión de elementos gramaticales, pues,
al partir del análisis exhaustivo de elementos previamente realizado
por la Real Academia Española, resulta totalmente innecesario definir
el proceso a llevar a cabo para efectuar dicho análisis. Así, se ignora
en el trabajo de investigación la relación existente entre la organización
de los morfemas en los elementos y los rasgos gramaticales que surgen
como producto de esta organización.
\item El nivel semántico utiliza una agrupación de roles en argumentos que,
aun lógica y válida a la vista, resulta aparentemente arbitraria en
cuanto a su selección. Dicha agrupación proviene, sin embargo, de
un intento de compatibilizar los argumentos definidos con los presentes
en los lexicones procedentes del corpus \textit{AnCora}, posteriormente
utilizados en la implementación de este nivel del modelo.
\end{enumerate}
\bigskip

Por último, a lo largo de toda la redacción que abarca el modelo a
continuación, se utiliza en las secciones una notación específica
para definir ciertos apartados o estructuras formalmente. Para evitar
cualquier tipo de confusión, y definirla en su totalidad en los casos
donde esta difiera de la norma, se mostrarán ejemplos y explicaciones
previas a su utilización en el trabajo.

\section{Nivel léxico-morfológico}

El nivel léxico-morfológico es el primer nivel a tratar en el modelo,
pues trata la base léxica sobre la que se contruyen, posteriormente,
las estructuras internas de los enunciados. Además, se describen todos
los posibles rasgos de las unidades gramaticales, con el fin de poder
utilizarlos para hacer posible su análisis en el nivel sintáctico
y en el nivel semántico. Para ello, se introduce la lexía como unidad
elemental sobre la que construir el resto del modelo.

\bigskip

La lexía es un elemento gramatical cuyas partes presentan un elevado
índice de inseparabilidad morfológica o forman una función distinta,
no necesariamente presentando cierto valor semántico, que cualquier
subconjunto de dichas partes. Las lexías engloban términos más convencionales
como la palabra y les aporta una definición menos problemática, pero
no se limitan únicamente a estos.

\bigskip

Según su estructura morfológica y visual, las lexías se pueden clasificar
de la siguiente manera:
\begin{enumerate}
\item \textit{Lexías simples.} Constan de un conjunto de morfemas carente
de espaciado o separación simbólica alguna. Se incluyen aquí tanto
formas simples como compuestas de las palabras, a diferencia de la
clasificación convencional de lexía, pues permite un procesado más
simple de las mismas.
\item \textit{Lexías complejas.} También denominadas como locuciones, se
componen de varios morfemas donde sí se presenta una separación mediante
espacios o símbolos. Cada isla de morfemas se podría considerar equivalente
a una lexía simple, sin embargo, la totalidad de la lexía aporta una
función y un valor semántico distintos a cada una de las lexías simples
discernibles en su interior, o cualquier agrupación de estas.
\item \textit{Lexías textuales.} Los nombres propios, los términos numéricos,
las expresiones hechas, las citaciones y cualquier término insondable
por las herramientas del ámbito gramatical se consideran lexías textuales,
pues difieren de las lexías simples y las lexías complejas en que
su significado proviene de una asignación artifical del mismo, de
un contexto externo al entorno o de una interpretación poética de
una o más lexías de otra clase.
\item \textit{Símbolos.} Estos son constituidos por cualquier marcación
indivisible pero estrictamente individual con una función exclusivamente
gramatical. Sus usos incluyen, pero no se limitan a, la delimitación
o la enumeración de elementos gramaticales, y no constan en ningún
alfabeto.
\end{enumerate}
\bigskip

Por otro lado, según su función y la presencia de valor semántico,
se pueden clasificar como se presenta a continuación, siendo esta
la clasificación de mayor utilidad a lo largo del modelo:
\begin{enumerate}
\item \textit{Lexías semánticas. }Se diferencian al poseer un valor semántico
y al constituir casi la totalidad del repertorio léxico. Los adjetivos,
los adverbios, los sustantivos y los verbos son los cuatro tipos de
lexías semánticas.
\item \textit{Lexías determinantes. }Tienen como función la de concretar
o determinar lexías, tornando conceptos en referencias. Además, pueden
aportar rasgos gramaticales adicionales a dichas lexías, o reforzar
los rasgos ya presentes en ellas.
\item \textit{Lexías gramaticales. }Estas cohesionan diversas lexías entre
sí, formando jerarquías y enumeraciones, o sirven para cambiar la
función de otras lexías. Se incluyen aquí las conjunciones, las preposiciones
y los símbolos.
\end{enumerate}

\subsection{Rasgos gramaticales}

Los rasgos gramaticales son propiedades inherentes de las lexías que
pueden tomar valores cuantizados y limitados que se encuentran definidos
según y para cada rasgo presente. Su presencia se encuentra mayoritariamente
condicionada por la clase de lexía en cuestión y, en ciertos casos,
por la subclase de la lexía.

\bigskip

Formalmente y para su nombramiento, se utiliza la siguiente nomenglatura:
{[}aaa{]}

\[
[\textsc{Rasgo}:\textsc{valor}]
\]


\subsection{Las lexías semánticas}

Adjetivos

Adverbios

Sustantivos

Verbos

\subsection{Las lexías determinantes}

Artículos

Cuantificadores

Demostrativos

Interrogativos

Numerales

Posesivos

Pronombres personales

Relativos

\subsection{Las lexías gramaticales}

Conjunciones

Preposiciones

Símbolos

\section{Nivel sintáctico}

\subsection{La enumeración}

\subsection{La cláusula}

\subsection{El sintagma}

El sintagma adjetival

El sintagma adverbial

El sintagma nominal

El sintagma verbal

\section{Nivel semántico}

\part{Implementación del modelo}

aaa pero AAA, asi que ªªª

\section{Accesibilidad al público}

\section{Evaluación del modelo}

\section{Conclusiones}

\printbibliography[heading=bibintoc]

\end{document}
